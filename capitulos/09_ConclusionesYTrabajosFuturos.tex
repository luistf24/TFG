\chapter{Conclusiones y trabajos futuros}
Como hemos podido ver a lo largo de la memoria, actualmente la mayoría de las aplicaciones utilizan las mismas herramientas salvo excepciones.\par
Todas usan un intercambio de claves \emph{Diffie-Hellman} salvo en \emph{iMessage} que usa el servicio de identidad \emph{IDS}. Esto puede llegar a ser peligroso ya que al estar todas las claves almacenadas en estos servidores, se podría vulnerar la seguridad de los mensajes con distintos ataques si se accede a la información alojada en estos.\par
Todas utilizan alguna función hash de la familia \emph{SHA-2} para la validación de los mensajes salvo Line. Esto es así ya que para cifrar los mensajes utiliza \emph{AES-256} con el modo \emph{GCM}. Este modo, que es el que he utilizado en la implementación, permite genera un \emph{salt} durante la encriptación del mensaje, que a la hora de desencriptarlo permite validarlo, ahorrando así el uso de herramientas externas para este proceso aumentando la velocidad y la seguridad de la aplicación. Ya que como hemos podido ver, las familias de funciones resumen van variando ya que a lo largo del tiempo van encontrando vulnerabilidades en estas. Al usar este modo, los creadores de la aplicación se ahorran tener que ir cambiado de función hash, con lo que ello conlleva, para seguir garantizando la seguridad de la aplicación. Esto fue lo que paso con el protocolo \emph{MTProto}, en la primera versión de este utilizaban la función resumen \emph{SHA-1} y al encontrarse una vulnerabilidad en esta, tuvo que cambiarse por la función \emph{SHA-256}.\par
En lo que coinciden todas es en usar \emph{AES-256} para cifrar los mensajes, si bien, no todas usan el mismo modo. Por ejemplo las aplicaciones con el protocolo \emph{Signal} utilizan el modo \emph{CBC} mientras que Line, que usa el protocolo \emph{Letter Sealing}, utiliza el modo \emph{GCM}. \par
Resulta destacable el hecho de que todas las aplicaciones vistas no ocultan el proceso criptográfico que siguen para cifrar los mensajes, generar las claves e intercambiarlas. Esto es así por razones de transparencia y seguridad. Ya que al revelar que criptosistemas utilizan permiten a los usuarios tener una idea de cómo se maneja la privacidad de sus comunicaciones y a los expertos revisar y auditar para identificar posibles vulnerabilidades en el cifrado.\par
Un hecho que probablemente ponga en jaque la seguridad de las aplicaciones actuales es la aparición de la computación cuántica.\\
Existen algoritmos como el de Peter Shor que permiten factorizar números muy grandes en factores primos en tiempo polinomial \cite{Shor1994} rompiendo así criptosistemas como \emph{RSA}. Si bien para poder romperlos se estima que la computadora cuántica tiene que tener millones de qubits y la computadora más potente que hay en la actualidad tiene 53, por lo que no hay un peligro inminente. Aun así hay alternativas, ya que con la aparición de la computación cuántica se creó una nueva rama llamada criptografía post cuántica centrada en la búsqueda de algoritmos resistentes a la computación cuántica. En la actualidad existen criptosistemas que son resistentes a estos algoritmos, un ejemplo de es el criptosistema de McEliece. Este algoritmo asimétrico fue desarrollado en 1978 por Robert McEliece y aunque en la actualidad no es muy utilizado debido al tamaño de claves que maneja, en un futuro podría empezar a extenderse debido a que es resistente a los algoritmos cuánticos de la actualidad.\par
Como trabajo futuro se podría ampliar el estudio incorporando nuevas aplicaciones que utilicen criptosistemas distintos además de ampliar la memoria explicando como las aplicaciones actuales permiten cifrar elementos distintos a los mensajes de texto como pueden ser las fotos, gifs, vídeos, mensajes de voz y llamadas entre otros. Además esto permitiría ampliar la implementación para incorporarlos y poder enviar estos archivos de manera segura a través de esta.\\
También podría resultar interesante realizar una implementación que utilice algún criptosistema post cuántico realizando un estudio comparativo sobre la eficiencia y seguridad con la implementación ya realizada. Viendo así la viabilidad de una aplicación comercial, que en un futuro, se pueda llegar a desarrollar garantizando este plus de seguridad que las aplicaciones actuales no aportan.
