\chapter{Criptografía y Curvas Elípticas:}
%\noindent\rule[-1ex]{\textwidth}{2pt}\\[4.5ex]

En este capítulo se introducirá la teoría sobre criptografía y curvas elípticas necesaria para entender la base detrás de los criptosistemas de las aplicaciones de mensajería.

\section{Introducción a la criptografía}
Mayormente la información de este apartado ha sido obtenida de \cite{GomezPardo2002b}
\subsection{Cifrado y secreto}

\begin{itemize}
	\item $\mathcal{M}$ el conjunto de los mensajes, textos en claro o \emph{plaintexts},
	\item $\mathcal{C}$ el conjunto de los criptogramas o \emph{cyphertexts},
	\item $\mathcal{K} \subseteq \mathcal{K}_p\times\mathcal{K}_s$ el espacio de clave o \emph{key space}
\end{itemize}
Un criptosistema viene definido por dos aplicaciones
$$E:\mathcal{K}_p\times\mathcal{M}\rightarrow\mathcal{C}$$
$$\mathcal{D}:\mathcal{K}_s\times\mathcal{C}\rightarrow\mathcal{M}$$
tales que para cualquier clave $k_p \in \mathcal{K}_p$, existe una clave $k_s$ de manera que dato cualquier mensaje $m \in \mathcal{M}$,
$$
\mathcal{D}(k_s,E(k_p,m))=m.
$$
Fijadas claves $k_p \in \mathcal{K}_p$ y sus correspondiente $k_s \in \mathcal{K}_s$ se definen las funciones de cifrado y descifrado como:
$$
	E_{k_p}:\mathcal{M}\rightarrow\mathcal{C},[E_{k_p}(m)=E(k_p,m)]
$$
$$
	D_{k_p}:\mathcal{C}\rightarrow\mathcal{M},[D_{k_s}(c)=D(k_s,c)]
$$
En la criptografía clásica, también llamada simétrica, se tiene que $\mathcal{K}_p=\mathcal{K}_s$ y $k_s = k_p = k \in \mathcal{K}$, o al menos hay métodos eficientes para conocer $k_s$ a partir de $k_p$ y viceversa. En la criptografía asimétrica, no se conocen métodos eficientes para conocer $k_s$ a partir de $k_p$.

\subsection{Objetivos de la criptografía}
\begin{itemize}
		\item \textbf{Confidencialidad:} La información solo puede ser accesible por las entidades autorizadas. 
		\item \textbf{Integridad:} La información no ha sido alterada en el envío.
		\item \textbf{Autenticidad:} La información proviene de quién afirma haberla enviado
		\item \textbf{No repudio:} El emisario de una información no puede a posteriori negar que se realizado tal envío.
\end{itemize}

\subsection{Ataques}
Se sigue el principio de Kerckhoffs el cual establece que el adversario conoce todos los detalles del criptosistema excepto la clave empleada. Los posibles ataques son:
\begin{itemize}
		\item \textbf{Criptograma} El adversario conoce el criptograma.
		\item \textbf{Mensaje Conocido} El atacante conoce parejas mensaje/criptograma cifradas con una misma clave
		\item \textbf{Mensaje escogido} El atacante puede generar criptogramas para mensajes de su elección. Una vez obtenidas dichas parejas, trata de averiguar el mensaje correspondiente a un criptograma desconocido.
		\item \textbf{Mensaje escogido-adaptativo} El atacante no solo puede generar pareas mensaje/criptograma a su elección, sino que puede hacerlo tantas veces como quiera realizando los análisis que considere oportunos
		\item \textbf{Criptograma escogido y escogido-adaptativo} Similar a los anteriores pero partiendo del criptograma, teniendo acceso a descifrar los criptogramas que desee, inicialmente o a lo largo del proceso. Lo que se busca en este ataque es la clave.
\end{itemize}
\newpage
\section{El algoritmo Rijndael AES}
El algoritmo Rijndael llamado así en honor a sus dos autores Joan Daemen y Vicent Rijmen, es un algoritmo de cifrado por bloques que fue adoptado en octubre de 2000 por el NIST(\emph{National Institute for Standards and Technology}) para su empleo en aplicaciones criptográficas no militares en sustitución del algoritmo \emph{DES} después de un proceso de más tres años en los que se buscaba un algoritmo que fuera potente, eficiente y fácil de implementar.\\
Está diseñado para manejar longitudes de clave y de bloque variables entre los 128 y los 256 bits y aunque estos sean variables, en el estándar adoptado por el Gobierno de Estados Unidos en 2001 \cite{aesUsa} establece una longitud fija de bloque de 128 bits y una longitud de clave a escoger entre 128, 192 y 256 bits.\\
La información para los siguientes apartados de AES la he obtenido de \cite{En2011}.

\subsection{Cifrados de bloque}
Son criptosistemas de clave simétrica en los que la longitud de los bloques y claves es fija.\\
Este criptosistema se define
$$
	E:\mathbb{B}^K\times\mathbb{B}^N\rightarrow \mathbb{B}^N,
$$
$$
	D:\mathbb{B}^K\times\mathbb{B}^N\rightarrow \mathbb{B}^N,
$$
Donde N es el tamaño del bloque y K es el tamaño de la clave.

\subsection{Estructura de AES}
En el algoritmo AES se define cada ronda como una composición de cuatro funciones invertibles diferentes, formando tres \emph{capas}. Estas funciones tienen un propósito específico:
\begin{itemize}
	\item \textbf{Capa de mezcla lineal:} Formada por las funciones \emph{DesplazarFila} y \emph{MezclarColumnas} y permite obtener un alto nivel de difusión a lo largo de varias rondas.
	\item \textbf{Capa no lineal:} Formada por la función \emph{ByteSub} y es la aplicación paralela de s-cajas con propiedades óptimas de no linealidad.
	\item \textbf{Capa de adición de clave:} Es un simple \emph{or-exclusivo} entre el estado intermedio y la subclave correspondiente a cada ronda.
\end{itemize}

\subsection{Elementos de AES}
AES es un algoritmo que se basa en aplicar un número determinado de rodas a un valor intermedio denominado \emph{estado} que puede ser representado por una matriz rectangular que posee cuatro filas y $N_{b}$ columnas. Análogamente la clave tiene la misma estructura, una matriz de cuatro filas y $N_{k}$.
El bloque ha cifrar o descifrar se traslada directamente byte a byte sobre la matriz de estado de columna en columna($a_{0,0}, a_{1,0}, a_{2,0}, a_{3,0}, a_{0,1} ...$)

\begin{table}[htb]
	\begin{center}
		\begin{tabular}{| l | l | l | l |}
				\hline
				$\math{a}_{0,0}$ & $\math{a}_{0,1}$ & $\math{a}_{0,2}$ & $\math{a}_{0,3}$\\ \hline
				$\math{a}_{1,0}$ & $\math{a}_{1,1}$ & $\math{a}_{1,2}$ & $\math{a}_{1,3}$\\ \hline
				$\math{a}_{2,0}$ & $\math{a}_{2,1}$ & $\math{a}_{2,2}$ & $\math{a}_{2,3}$\\ \hline
				$\math{a}_{3,0}$ & $\math{a}_{3,1}$ & $\math{a}_{3,2}$ & $\math{a}_{3,3}$\\ \hline
		\end{tabular}
		\caption{Ejemplo de matriz de estado con $N_b=4$(128 bits).}
	\end{center}
\end{table}

\begin{table}[htb]
	\begin{center}
		\begin{tabular}{| l | l | l | l |}
				\hline
				$\math{k}_{0,0}$ & $\math{k}_{0,1}$ & $\math{k}_{0,2}$ & $\math{k}_{0,3}$\\ \hline
				$\math{k}_{1,0}$ & $\math{k}_{1,1}$ & $\math{k}_{1,2}$ & $\math{k}_{1,3}$\\ \hline
				$\math{k}_{2,0}$ & $\math{k}_{2,1}$ & $\math{k}_{2,2}$ & $\math{k}_{2,3}$\\ \hline
				$\math{k}_{3,0}$ & $\math{k}_{3,1}$ & $\math{k}_{3,2}$ & $\math{k}_{3,3}$\\ \hline
		\end{tabular}
		\caption{Ejemplo de clave con $N_k=4$(128 bits).}
	\end{center}
\end{table}

En otros casos el bloque y la clave pueden ser representados como vectores de registro de 32 bits donde cada registro esta compuesto por los bytes de la columna correspondiente ordenados en orden descendiente.\\

Siendo $B$ el bloque que queremos cifrar y $M$ la matriz de estado, el algoritmo AES con $n$ quedaría:

\begin{enumerate}
	\item Calcular $K_0, K_1,...,K_n$ subclaves a partar de la clave $K$.
	\item $M\leftarrow B \oplus K_0$
	\item Para $i=1$ hasta $n$ hacer
	\begin{description}
			\item Aplicar la roda \emph{i}-ésima del algoritmo con la subclave $K_i$
	\end{description}
\end{enumerate}
Como las funciones usadas en cada ronda son invertibles, para descifrar aplicaremos las funciones inversas de las funciones usadas para cifrar en el orden opuesto.

\newpage
\begin{table}[htb]
	\begin{center}
		\begin{tabular}{| l | l | l | l |}
				\hline
				& $N_b = 4$(128 bits) & $N_b = 6$(192 bits)& $N_b = 8$(256 bits)\\ \hline
				$N_b = 4$(128 bits)& 10 & 12 & 14\\ \hline
				$N_b = 6$(128 bits)& 12 & 12 & 14\\ \hline
				$N_b = 8$(128 bits)& 14 & 14 & 14\\ \hline
		\end{tabular}
		\caption{Número de rodas en función del tamaño de la clave y bloque}
		\label{rondas_aes}
	\end{center}
\end{table}

\subsection{Las Rondas de AES}
Dado que el algoritmo AES puede aplicarse para longitudes diferentes de bloque y clave, el número de rondas es variables, como se ha visto en \ref{rondas_aes}.\\
Siendo $M$ la matriz de estado y $K_i$ la subclave correspondiente a la ronda $i$-ésima, cada ronda posee esta estructura:
\begin{enumerate}
	\item $M \leftarrow ByteSub(M)$
	\item $M \leftarrow DesplazarFila(M)$
	\item $M \leftarrow MezclarColumnas(M)$
	\item $M \leftarrow K_i \oplus M$
\end{enumerate}
En la última ronda se hacen solo los tres primeros pasos del algoritmo.

\begin{description}
	\item \textbf{Función ByteSub}\\
		La función \emph{ByteSub} es una sustitución no lineal que se aplica a cada byte de la matriz de estado. Se obtiene componiendo dos transformaciones:
		\begin{enumerate}
			\item Cada byte se considera como un elemento del $GF(2^8)$ generado por el polinomio irreducible $m(x)=x^8+x^4+x^3+x+1$ y es sustituido por su inversa multiplicativa quedando el valor cero inalterado. 
			\item A continuación se aplica la siguiente transformación afín en $GF(2)$ siendo $x_0, x_1,...,x_7$ los bits del byte correspondiente e $y_0, y_1,...,y_7$ los del resultado:

				\begin{equation*} 
					\begin{bmatrix} 
						y_0\\
						y_1\\
						y_2\\
						y_3\\
						y_4\\
						y_5\\
						y_6\\
						y_7\\
					\end{bmatrix}
					=
					\begin{bmatrix} % O matrices como esta de 4 x 3
						1 & 0 & 0 & 0 & 1 & 1 & 1 & 1\\
						1 & 1 & 0 & 0 & 0 & 1 & 1 & 1\\
						1 & 1 & 1 & 0 & 0 & 0 & 1 & 1\\
						1 & 1 & 1 & 1 & 0 & 0 & 0 & 1\\
						1 & 1 & 1 & 1 & 1 & 0 & 0 & 0\\
						0 & 1 & 1 & 1 & 1 & 1 & 0 & 0\\
						0 & 0 & 1 & 1 & 1 & 1 & 1 & 0\\
						0 & 0 & 0 & 1 & 1 & 1 & 1 & 1\\
					\end{bmatrix}
					\begin{bmatrix}
						x_0\\
						x_1\\
						x_2\\
						x_3\\
						x_4\\
						x_5\\
						x_6\\
						x_7\\
					\end{bmatrix}
					+
					\begin{bmatrix}
						1\\
						1\\
						0\\
						0\\
						0\\
						1\\
						1\\
						0\\
					\end{bmatrix}
			\end{equation*}
		\end{enumerate}
\end{description}
