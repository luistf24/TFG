\chapter{Criptografía y Curvas Elípticas:}
%\noindent\rule[-1ex]{\textwidth}{2pt}\\[4.5ex]

En este capítulo se introducirá la teoría sobre criptografía y curvas elípticas necesaria para entender la base detrás de los criptosistemas de las aplicaciones de mensajería.
Mayormente la información ha sido obtenida de \cite{GomezPardo2002b}

\section{Cifrado y secreto}

\begin{itemize}
	\item $\mathcal{M}$ el conjunto de los mensajes, textos en claro o \emph{plaintexts},
	\item $\mathcal{C}$ el conjunto de los criptogramas o \emph{cyphertexts},
	\item $\mathcal{K} \subseteq \mathcal{K}_p\times\mathcal{K}_s$ el espacio de clave o \emph{key space}
\end{itemize}
Un criptosistema viene definido por dos aplicaciones
$$\mathcal{E}:\mathcal{K}_p\times\mathcal{M}\rightarrow\mathcal{C}$$
$$\mathcal{D}:\mathcal{K}_s\times\mathcal{C}\rightarrow\mathcal{M}$$
tales que para cualquier clave $k_p \in \mathcal{K}_p$, existe una clave $k_s$ de manera que dato cualquier mensaje $m \in \mathcal{M}$,
$$
\mathcal{D}(k_s,E(k_p,m))=m.
$$
Fijadas claves $k_p \in \mathcal{K}_p$ y sus correspondiente $k_s \in \mathcal{K}_s$ se definen las funciones de cifrado y descifrado como:
$$
	E_{k_p}:\mathcal{M}\rightarrow\mathcal{C},[E_{k_p}(m)=E(k_p,m)]
$$
$$
	D_{k_p}:\mathcal{C}\rightarrow\mathcal{M},[D_{k_s}(c)=D(k_s,c)]
$$
En la criptografía clásica, también llamada simétrica, se tiene que $\mathcal{K}_p=\mathcal{K}_s$ y $k_s = k_p = k \in \mathcal{K}$, o al menos hay métodos eficientes para conocer $k_s$ a partir de $k_p$ y viceversa. En la criptografía asimétrica, no se conocen métodos eficientes para conocer $k_s$ a partir de $k_p$.

\section{Objetivos de la criptografía}
\begin{itemize}
		\item \textbf{Confidencialidad:} La información solo puede ser accesible por las entidades autorizadas. 
		\item \textbf{Integridad:} La información no ha sido alterada en el envío.
		\item \textbf{Autenticidad:} La información proviene de quién afirma haberla enviado
		\item \textbf{No repudio:} El emisario de una información no puede a posteriori negar que se realizado tal envío.
\end{itemize}
