\chapter{Criptografía Simétrica. AES}
\label{chap:cuatro}

En este capítulo voy a explicar el cifrado Rijndael AES el cual es un cifrado de bloque simétrico muy utilizado actualmente por aplicaciones como \emph{Telegram}, \emph{WhatsApp} y \emph{FacebookChat} entre otras. Este algoritmo se usa principalmente para cifrar los mensajes ya que es muy eficiente a nivel computacional.
\section{El algoritmo Rijndael AES}
El algoritmo Rijndael llamado así en honor a sus dos autores Joan Daemen y Vicent Rijmen, es un algoritmo de cifrado por bloques que fue adoptado en octubre de 2000 por el NIST (\emph{National Institute for Standards and Technology}). Se desarrolló con la finalidad de sustituir el algoritmo \emph{DES} después de un proceso de estandarización que duró alrededor de 5 años.

Está diseñado para manejar claves y bloques con una longitud variable entre los 128 y los 256 bits. Aunque las longitudes sean variables, en el estándar que adoptó el Gobierno de Estados Unidos en 2001 \cite{aesUsa} establece una longitud fija de bloque de 128 bits y una longitud de clave entre 128, 192 y 256 bits.

La información para los siguientes apartados de AES la he obtenido de \cite{En2011} y de \cite{criptografia}.
\newpage
\section{Estructura de AES}
AES es un algoritmo que se basa en aplicar un número determinado de rondas a un valor intermedio denominado \emph{estado} que puede verse como una matriz rectangular que posee cuatro filas y $N_{b}$ columnas. Análogamente, la clave tiene la misma estructura.

Cada byte del bloque que se quiera cifrar o descifrar se va almacenando en la matriz de estados por columnas, es decir, el primer byte iría a la posición $a_{0,0}$, el segundo byte iría a la posición $a_{1,0}$ y así sucesivamente. De la misma manera se procedería con la clave.

\begin{table}[htb]
	\begin{center}
		\begin{tabular}{| c | c | c | c |}
				\hline
				$\math{a}_{0,0}$ & $\math{a}_{0,1}$ & $\math{a}_{0,2}$ & $\math{a}_{0,3}$\\ \hline
				$\math{a}_{1,0}$ & $\math{a}_{1,1}$ & $\math{a}_{1,2}$ & $\math{a}_{1,3}$\\ \hline
				$\math{a}_{2,0}$ & $\math{a}_{2,1}$ & $\math{a}_{2,2}$ & $\math{a}_{2,3}$\\ \hline
				$\math{a}_{3,0}$ & $\math{a}_{3,1}$ & $\math{a}_{3,2}$ & $\math{a}_{3,3}$\\ \hline
		\end{tabular}
		\caption{Ejemplo de matriz de estado con $N_b=4$(128 bits).}
	\end{center}
\end{table}

\begin{table}[htb]
	\begin{center}
		\begin{tabular}{| c | c | c | c |}
				\hline
				$\math{k}_{0,0}$ & $\math{k}_{0,1}$ & $\math{k}_{0,2}$ & $\math{k}_{0,3}$\\ \hline
				$\math{k}_{1,0}$ & $\math{k}_{1,1}$ & $\math{k}_{1,2}$ & $\math{k}_{1,3}$\\ \hline
				$\math{k}_{2,0}$ & $\math{k}_{2,1}$ & $\math{k}_{2,2}$ & $\math{k}_{2,3}$\\ \hline
				$\math{k}_{3,0}$ & $\math{k}_{3,1}$ & $\math{k}_{3,2}$ & $\math{k}_{3,3}$\\ \hline
		\end{tabular}
		\caption{Ejemplo de clave con $N_k=4$(128 bits).}
	\end{center}
\end{table}

En algunas operaciones que se llevan a cabo, interesa ver los bloques y las claves como vectores de registro de 32 bits.

Siendo $B$ el bloque que queremos cifrar y $S$ la matriz de estado, el algoritmo AES con $n$ rondas se resume en:

\begin{enumerate}
	\item Calcular $K_0, K_1,...,K_n$ subclaves a partir de la clave $K$.
	\item $S\leftarrow B \oplus K_0$.
	\item Para $i=1$ hasta $n$ hacer:
	\begin{description}
			\item Aplicar la ronda \emph{i}-ésima del algoritmo con la subclave $K_i$.
	\end{description}
\end{enumerate}

Para descifrar únicamente habría que repetir el algoritmo en orden inverso e invirtiendo las funciones.

\begin{table}[htb]
	\begin{center}
		\begin{tabular}{| c | c | c | c |}
				\hline
				& $N_b = 4$(128 bits) & $N_b = 6$(192 bits)& $N_b = 8$(256 bits)\\ \hline
				$N_k = 4$(128 bits)& 10 & 12 & 14\\ \hline
				$N_k = 6$(192 bits)& 12 & 12 & 14\\ \hline
				$N_k = 8$(256 bits)& 14 & 14 & 14\\ \hline
		\end{tabular}
		\caption{Número de rondas en función del tamaño de la clave y bloque.}
		\label{rondas_aes}
	\end{center}
\end{table}

%En el algoritmo AES se define cada ronda como una composición de cuatro funciones invertibles diferentes. Estas funciones se pueden agur.
%\begin{itemize}
	%\item \textbf{Capa de mezcla lineal:} formada por las funciones \emph{DesplazarFila} y \emph{MezclarColumnas} que permite obtener un alto nivel de difusión a lo largo de varias rondas.
	%\item \textbf{Capa no lineal:} formada por la función \emph{ByteSub} y es la aplicación paralela de s-cajas con propiedades óptimas de no linealidad.
	%\item \textbf{Capa de adición de clave:} es un simple \emph{o-exclusivo} entre el estado intermedio y la subclave correspondiente a cada ronda.
%\end{itemize}

%\newpage
\section{Las Rondas de AES}
Una vez visto algunas de las propiedades de los cuerpos finitos, ya disponemos de las herramientas necesarias para describir las rondas de AES.
Dado que este algoritmo puede aplicarse para longitudes diferentes de bloque y clave, el número de rondas es variable, como se ha visto en \ref{rondas_aes}.

Siendo $S$ la matriz de estado y $K_i$ la subclave correspondiente a la ronda $i$-ésima, cada ronda posee esta estructura:
\begin{enumerate}
	\item $S \leftarrow ByteSub(S)$,
	\item $S \leftarrow DesplazarFila(S)$,
	\item $S \leftarrow MezclarColumnas(S)$,
	\item $S \leftarrow AddRoundKey(S) = K_i \oplus S$.
\end{enumerate}
En la última ronda no se ejecuta el tercer paso del algoritmo.

	\subsubsection{ByteSub}
		La función \emph{ByteSub} es una sustitución no lineal. Se obtiene componiendo dos transformaciones:
		\begin{enumerate}
			\item Cada byte se considera como un elemento del $\operatorname{GF}(2^8)$ generado por el polinomio irreducible $a(x)=x^8+x^4+x^3+x+1$ y es sustituido por su inversa multiplicativa quedando el valor cero inalterado. 
			\item A continuación se aplica la siguiente transformación afín en $\operatorname{GF}(2^8)$ siendo $x_0, x_1,...,x_7$ los bits del byte correspondiente del bloque e $y_0, y_1,...,y_7$ la salida de la operación.

				\begin{equation*} 
					\begin{bmatrix} 
						y_0\\
						y_1\\
						y_2\\
						y_3\\
						y_4\\
						y_5\\
						y_6\\
						y_7\\
					\end{bmatrix}
					=
					\begin{bmatrix} % O matrices como esta de 4 x 3
						1 & 0 & 0 & 0 & 1 & 1 & 1 & 1\\
						1 & 1 & 0 & 0 & 0 & 1 & 1 & 1\\
						1 & 1 & 1 & 0 & 0 & 0 & 1 & 1\\
						1 & 1 & 1 & 1 & 0 & 0 & 0 & 1\\
						1 & 1 & 1 & 1 & 1 & 0 & 0 & 0\\
						0 & 1 & 1 & 1 & 1 & 1 & 0 & 0\\
						0 & 0 & 1 & 1 & 1 & 1 & 1 & 0\\
						0 & 0 & 0 & 1 & 1 & 1 & 1 & 1\\
					\end{bmatrix}
					\begin{bmatrix}
						x_0\\
						x_1\\
						x_2\\
						x_3\\
						x_4\\
						x_5\\
						x_6\\
						x_7\\
					\end{bmatrix}
					+
					\begin{bmatrix}
						1\\
						1\\
						0\\
						0\\
						0\\
						1\\
						1\\
						0\\
					\end{bmatrix}
			\end{equation*}
		\end{enumerate}

		La función inversa de $ByteSub$ es la aplicación inversa de cada byte de la matriz de estado.

	\subsubsection{DesplazarFila}
		Esta función desplaza a la izquierda de manera cíclica las filas de la matriz de estado. Cada fila $f_i$ se desplaza un número de posiciones $c_i$ diferente. Mientras que $c_0$ siempre es igual a cero, el resto de valores vine en función de $N_b$ como se puede ver en \ref{ciennb}.

		La función inversa será el desplazamiento de las filas de la matriz el mismo número de posiciones pero en el sentido contrario.

		\begin{table}[htb]
			\begin{center}
				\begin{tabular}{| c | c | c | c |}
						\hline
						$N_b$ & $c_1$ & $c_2$ & $c_3$\\ \hline
						4 & 1 & 2 & 3\\ \hline 
						6 & 1 & 2 & 3\\ \hline 
						8 & 1 & 3 & 4\\ \hline 
				\end{tabular}
				\caption{Valores de $c_i$ según el tamaño de bloque $N_b$}
				\label{ciennb}
			\end{center}
		\end{table}

		\begin{figure}[htb]
			\centering
			\includegraphics[scale=0.4]{imagenes/aesdesplazarmezclar.png} 
			\caption{Esquema de las funciones $MezclarColumnas$ y $DesplazarFila$ \cite{En2011}}
			\label{desplazarymezclar}
		\end{figure}
%\newpage
	\subsubsection{MezclarColumnas}
	Durante la aplicación de esta función cada columna del vector de estado es vista como una matriz $4 \times 1$ donde sus coeficientes pertenecen a $\mathbb{F}_{256}$. Aplicar MezclarColumnas a cada estado equivale a multiplicar cada columna por la matriz $4 \times 4$.

			\begin{center}
					\begin{pmatrix} 
						02 & 03 & 01 & 01 \\
						01 & 02 & 03 & 01 \\
						01 & 01 & 02 & 03 \\
						03 & 01 & 01 & 02 \\
					\end{pmatrix}
			\end{center}

\section{Cálculo de las Subclaves}
Las subclaves $K_i$ se obtienen de la clave principal $K$ a partir de dos funciones. Una función de expansión y otra de selección.

La función de expansión obtiene a partir del valor de $K$ una secuencia de $4(n+1)N_b$ bytes, donde $n$ es el número de rondas.

La función de selección toma de la secuencia obtenida bloques y los asigna a cada $K_i$.

Sea $K(i)$ un vector de bytes de tamaño $4N_k$ conteniendo la clave y sea $W(i)$ un vector de $N_b(n+1)$ registros de 4 bytes, siendo $n$ el número de rondas. 
La función de expansión tiene dos versiones según el valor de $N_k$:

\begin{itemize}
	\item Si $N_k\le6$:
	\begin{algorithm}
		Para $i$ desde 0 hasta $N_{k}-1$ hacer:
		\begin{description}
			$W(i)\leftarrow(K(4·i), K(4·i+1), K(4·i+2), K(4·i+3))$
		\end{description}
		Para $i$ desde $N_k$ hasta $N_{b}·(n+1)$ hacer:\\
			\hspace*{20}$tmp\leftarrow W(i-1)$\\
			\hspace*{20}Si $i \mod N_k = 0$\\
			\hspace*{40}$tmp\leftarrow Sub(Rot(tmp))\oplus Rc(i/N_k)$\\
			\hspace*{20}$W(i)\leftarrow W(i-N_k)\oplus tmp$
	\end{algorithm}
%\newpage
	\item Si $N_k>6$:
	\begin{algorithm}
		Para $i$ desde 0 hasta $N_{k}-1$ hacer:
		\begin{description}
			$W(i)\leftarrow(K(4·i), K(4·i+1), K(4·i+2), K(4·i+3))$
		\end{description}
		Para $i$ desde $N_k$ hasta $N_{b}·(n+1)$ hacer:\\
			\hspace*{20}$tmp\leftarrow W(i-1)$\\
			\hspace*{20}Si $i \mod N_k = 0$\\
			\hspace*{40}$tmp\leftarrow Sub(Rot(tmp))\oplus Rc(i/N_k)$\\
			\hspace*{20}Si $i \mod N_k = 4$\\
			\hspace*{40}$tmp\leftarrow Sub(tmp)$\\
			\hspace*{20}$W(i)\leftarrow W(i-N_k)\oplus tmp$
	\end{algorithm}
\end{itemize}
\newpage
La función \emph{Sub} devuelve el resultado de aplicar la s-caja de AES a cada uno de los bytes del registro de cuatro que se le pasa como parámetro, la función \emph{Rot} desplaza a la izquierda los bytes del registro y \emph{Rc(j)} es una constante que se define como:
\begin{itemize}
	\item $Rc(j)=(R(j),0,0,0)$.
	\item Cada $R(i)$ es el elemento de $\operatorname{GF}(2^8)$ correspondiente al valor $x^{i-1} \mod x^8+x^4+x^3+x+1$.
\end{itemize}

