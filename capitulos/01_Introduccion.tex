\chapter{Introducción}
Vivimos en una época en la que no se pueden concebir las relaciones sociales sin pensar en las redes sociales y en particular las aplicaciones de mensajería. Estas nos permiten conectarnos unos con otros independientemente de las barreras físicas.
Hay aplicaciones como  \emph{WhatsApp}, \emph{Facebook Messenger} o \emph{Telegram} que tienen 2.000, 931 y 700 millones de usuarios respectivamente, lo que supone un porcentaje significativo de la población mundial que usa aplicaciones de mensajería.\\  
Debido a esta enorme cantidad de usuarios las aplicaciones tienen que garantizar su seguridad y la privacidad. Es por esto que la criptografía ha cobrado un papel fundamental en la actualidad ya que las herramientas que ofrece son las que permiten garantizar dicha seguridad y privacidad de los usuarios de las aplicaciones. 

\section{Contexto histórico}
Las aplicaciones de mensajería aparecieron en la década de 1970. Una de las primeras fue el sistema \emph{PLATO}, este era una aplicación de asistencia para la computadora basada en un sistema informático de tiempo compartido por usuarios y programadores. Fue diseñada por Bitzer con la finalidad de hacer realidad el objetivo de educar por el ordenador y entre una de sus funcionalidades había un chat para que los usuarios se comunicaran entre sí de manera local.\\ 
Entre las décadas de 1980 y 1990 apareció la aplicación \emph{TALK}, esta fue diseñada para dispositivos con sistema operativos basados en \emph{UNIX/LINUX}. Esta aplicación permitía enviar mensajes entre usuarios a través de Internet. Si bien al principio solo permitía comunicarse entre usuarios que estuvieran conectados al mismo dispositivo aunque luego se amplió la funcionalidad permitiendo el envío de mensajes entre usuarios de otros sistemas.\\
Hasta 1996 no apareció una aplicación de mensajería que se pudiera usar en otros dispositivos con distintos sistemas operativos. 
Esta fue \emph{ICQ} y supuso un antes y un después ya que fue la primera en abarcar tantos usuarios y además añadió nuevas funcionalidades. En su momento de mayor popularidad alcanzó los 38 millones de usuarios, permitiendo atisbar el potencial de las aplicaciones de mensajería como medio de comunicación.
Esta aplicación añadía nuevas funcionalidades como eran un perfil de usuario personalizable, estado de conexión, emoticonos, transferencia de contactos, transferencia de archivos y chat grupales que fueron adoptadas por las nuevas aplicaciones de mensajería manteniéndose muchas de ellas hasta hoy en día.\\
A partir de esta empezaron a surgir nuevas aplicaciones de mensajería con mayor frecuencia, estas aplicaciones usaban cada una un protocolo distinto por lo que se llevó a los usuarios a tener distintos clientes para cada aplicación.
Algunas de las aplicaciones más populares que aparecieron en esta época fueron \emph{MSN Messenger} y \emph{AIM}. La más popular fue \emph{AIM} que en 2006 tenía el control del 52 por ciento del mercado de las aplicaciones de mensajería. \emph{MSN} necesito más años para ser más popular y hasta 2005 que no alcanzó su mayor pico llegando a atraer alrededor de 330 millones de usuarios activos cada mes. Lo hizo bajo el nombre de \emph{Windows Live Messenger}.\\ 
Para compensar el creciente número de protocolo surgieron aplicaciones multiclientes que permitían soportar varios de estos protocolos, algunas de estas fueron \emph{Pidgin} o \emph{Trillian}. Ambas aplicaciones permitían comunicarse usando protocolos como \emph{MSN}, \emph{MySpaceIM}, \emph{XMPP/Jabber(Google Talk, Facebook Messenger)} y \emph{Yahoo!} entre otros.\\ 
A la vez se popularizaron las videollamadas por lo que aparecieron nuevas aplicaciones para aprovechar el nuevo nicho. Una de las primeras en aparecer fue \emph{Microsoft NetMeeting} aunque poco después apareció \emph{Skype} y se apropió de la mayoría de los usuarios de esta.\\
En 2010 los desarrolladores cambiaron de plataforma y dejaron de desarrollar aplicaciones de mensajería para ordenador para centrarse en los \emph{Smartphones}. Aparecieron aplicaciones como \emph{WhatsApp, Telegram} y \emph{Facebook Messenger} que como hemos visto al principio del capítulo, son fundamentales hoy en día y abarcan miles de millones de usuarios.

\section{Descripción del problema}
En esta memoria el problema que he abordado ha sido el desarrollo de un estudio teórico de los criptosistemas que utilizan las aplicaciones de mensajería más utilizadas en la actualidad. Para ello se desarrolla de manera rigurosa de las herramientas matemáticas e informáticas que utilizan estas. Este desarrollo abarca una introducción a la criptografía simétrica y asimétrica así como teoría de cuerpos finitos necesarios para entender adecuadamente las herramientas.
Un desarrollo en profundidad de los cifrados de bloque y del cifrado \emph{AES} en particular explicación exhaustiva de \emph{RSA} y del \emph{Problema del Logaritmo discreto} y como resultado de este, el intercambio de claves \emph{Diffie-Hellman}. Posteriormente he introducido la teoría de Curvas Elípticas necesaria para entender el análogo de las herramientas anteriores utilizando este cuerpo.
Después he introducido las \emph{funciones hash}, como construirlas usando la construcción de \emph{Merkle-Damgård} y las familias de funciones más utilizadas en las aplicaciones de mensajería actuales. A continuación he realizado una descripción como incorporan las aplicaciones de mensajería más populares los criptosistemas vistos anteriormente. Por último he desarrollado una aplicación de mensajería en la cual he aplicado lo visto previamente en la memoria \emph{Desarrollar esto más extensamente cuando tenga la aplicación}. 

\section{Técnicas utilizadas}
Para poder llevar a cabo esta memoria he necesitado de diversas herramientas matemáticas e informáticas que he ido viendo a lo largo de mi paso por los grados de matemáticas e informática.\\ 
Las herramientas matemáticas utilizadas han sido las siguientes.
\begin{itemize}
	\item Teoría de números: Función phi de Euler, Teorema Chino del Resto, Teorema de Euler y Teorema pequeño de Fermat.
	\item Teoría de cuerpos: Cuerpos finitos.
	\item Teoría de grupos: El problema de Logaritmo Discreto. 
	\item Teoría de Galois: Resultados de Cuerpos finitos.
	\item Geometría lineal: Transformaciones lineales.
	\item Geometría algebraica: Curvas elípticas y resultados de estas.
\end{itemize}

Y las herramientas informáticas
\begin{itemize}
	\item Criptografía: Criptosistema, criptografía simétrica, criptografía asimétrica, funciones hash, firma digital e intercambio de claves \emph{Diffie-Hellman}. 
	\item Algorítmica: Análisis de algoritmos, en casi todas los capítulos de la memoria aparece algún algoritmo y un análisis de este es necesario para su correcta comprensión.
\end{itemize}

		
