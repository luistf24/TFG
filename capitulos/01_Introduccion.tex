\chapter{Introducción}
Vivimos en una época en la que no se pueden concebir las relaciones sociales sin pensar en las redes sociales y en particular las aplicaciones de mensajería. Estas nos permiten conectarnos unos con otros independientemente de las barreras físicas.
Hay aplicaciones como  \emph{WhatsApp}, \emph{Facebook Messenger} o \emph{Telegram} que tienen 2.000, 931 y 700 millones de usuarios respectivamente, lo que supone un porcentaje significativo de la población mundial que usa aplicaciones de mensajería.\\  
Debido a esta enorme cantidad de usuarios las aplicaciones tienen que garantizar su seguridad y la privacidad. Es por esto que la criptografía ha cobrado un papel fundamental en la actualidad ya que las herramientas que ofrece son las que permiten garantizar dicha seguridad y privacidad de los usuarios de las aplicaciones. 

\section{Contexto histórico}
Las aplicaciones de mensajería aparecieron en la década de 1970. Una de las primeras fue el sistema \emph{PLATO} \emph{Desarrollar esto}. Entre las décadas de 1980 y 1990 apareció la aplicación \emph{TALK}, esta fue diseñada para dispositivos con sistema operativos basados en \emph{UNIX/LINUX} que permitía comunicarse a través de Internet \emph{Desarrollar esto}. Hasta 1996 no apareció una aplicación de mensajería que se pudiera usar en otros dispositivos con distintos sistemas operativos. Esta fue \emph{ICQ} \emph{Desarrollar esto}. A partir de estas ya empezaron a surgir nuevas aplicaciones, estas aplicaciones usaban cada una un protocolo distinto por lo que se llevó a los usuarios a tener distintos clientes para cada aplicación. Para compensar esto surgieron aplicaciones multiclientes que permitían soportar varios de estos protocolos, algunas de estas fueron \emph{Pidgin} o \emph{Trillian}. \emph{Continuar con la década del 2000}.


\section{Descripción del problema}

\section{Técnicas utilizadas}
