\documentclass[a4paper,11pt]{book}
\usepackage{chngpage}
%\documentclass[a4paper,twoside,11pt,titlepage]{book}
\usepackage{listings}
\usepackage{amssymb}
%\usepackage{amsmath}
\usepackage[utf8]{inputenc}
\usepackage[spanish]{babel}
\usepackage{array}
\usepackage{amsmath}
\usepackage{url}
\usepackage{graphicx}
\usepackage{subfig}
% \usepackage[style=list, number=none]{glossary} %
%\usepackage{titlesec}
%\usepackage{pailatino}

\decimalpoint
\usepackage{dcolumn}
\newcolumntype{.}{D{.}{\esperiod}{-1}}
\makeatletter
\addto\shorthandsspanish{\let\esperiod\es@period@code}
\makeatother


\usepackage[chapter]{algorithm}
\RequirePackage{verbatim}
%\RequirePackage[Glenn]{fncychap}
\usepackage{fancyhdr}
\usepackage{graphicx}
\usepackage{afterpage}

\usepackage{longtable}

\usepackage[pdfborder={000}]{hyperref} %referencia

\tolerance=1
\emergencystretch=\maxdimen
\hyphenpenalty=10000
\hbadness=10000

% ********************************************************************
% Re-usable information
% ********************************************************************
\newcommand{\myTitle}{Criptosistemas en aplicaciones de mensajería\xspace}
\newcommand{\myDegree}{Doble grado en Ingeniería Informática y Matemáticas\xspace}
\newcommand{\myName}{Luis Tormo Fabios\xspace}
\newcommand{\myProf}{Pedro A. García Sánchez\xspace}
%\newcommand{\myOtherProf}{Nombre Apllido1 Apellido2 (tutor2)\xspace}
%\newcommand{\mySupervisor}{Put name here\xspace}
\newcommand{\myFaculty}{Escuela Técnica Superior de Ingenierías Informática y de
Telecomunicación\xspace}
\newcommand{\myFacultyShort}{E.T.S. de Ingenierías Informática y de
Telecomunicación y Facultad de Ciencias\xspace}
\newcommand{\myDepartment}{Departamento de ...\xspace}
\newcommand{\myUni}{\protect{Universidad de Granada}\xspace}
\newcommand{\myLocation}{Granada\xspace}
\newcommand{\myTime}{\today\xspace}
\newcommand{\myVersion}{Version 0.1\xspace}


\hypersetup{
pdfauthor = {\myName (email (en) ugr ()},
pdftitle = {\myTitle},
pdfsubject = {},
pdfkeywords = {palabra_clave1, palabra_clave2, palabra_clave3, ...},
pdfcreator = {LaTeX con el paquete ....},
pdfproducer = {pdflatex}
}

%\hyphenation{}


%\usepackage{doxygen/doxygen}
%\usepackage{pdfpages}
\usepackage{url}
\usepackage{colortbl,longtable}
\usepackage[stable]{footmisc}
%\usepackage{index}

%\makeindex
%\usepackage[style=long, cols=2,border=plain,toc=true,number=none]{glossary}
% \makeglossary

% Definición de comandos que me son tiles:
%\renewcommand{\indexname}{Índice alfabético}
%\renewcommand{\glossaryname}{Glosario}

\pagestyle{fancy}
\fancyhf{}
\fancyhead[LO]{\leftmark}
\fancyhead[RE]{\rightmark}
\fancyhead[RO,LE]{\textbf{\thepage}}
\renewcommand{\chaptermark}[1]{\markboth{\textbf{#1}}{}}
\renewcommand{\sectionmark}[1]{\markright{\textbf{\thesection. #1}}}

\setlength{\headheight}{1.5\headheight}

\newcommand{\HRule}{\rule{\linewidth}{0.5mm}}
%Definimos los tipos teorema, ejemplo y definición podremos usar estos tipos
%simplemente poniendo \begin{teorema} \end{teorema} ...
\newtheorem{teorema}{Teorema}[chapter]
\newtheorem{proposicion}[teorema]{Proposición}
\newtheorem{ejemplo}[teorema]{Ejemplo}
\newtheorem{definicion}[teorema]{Definición}
\newenvironment{proof}{\paragraph{\emph{Demostración}\\}}{\hfill$\square$}

\definecolor{gray97}{gray}{.97}
\definecolor{gray75}{gray}{.75}
\definecolor{gray45}{gray}{.45}
\definecolor{gray30}{gray}{.94}

\lstset{ frame=Ltb,
     framerule=0.5pt,
     aboveskip=0.5cm,
     framextopmargin=3pt,
     framexbottommargin=3pt,
     framexleftmargin=0.1cm,
     framesep=0pt,
     rulesep=.4pt,
     backgroundcolor=\color{gray97},
     rulesepcolor=\color{black},
     %
     stringstyle=\ttfamily,
     showstringspaces = false,
     basicstyle=\scriptsize\ttfamily,
     commentstyle=\color{gray45},
     keywordstyle=\bfseries,
     %
     numbers=left,
     numbersep=6pt,
     numberstyle=\tiny,
     numberfirstline = false,
     breaklines=true,
   }
 
% minimizar fragmentado de listados
\lstnewenvironment{listing}[1][]
   {\lstset{#1}\pagebreak[0]}{\pagebreak[0]}

\lstdefinestyle{CodigoC}
   {
	basicstyle=\scriptsize,
	frame=single,
	language=C,
	numbers=left
   }
\lstdefinestyle{CodigoC++}
   {
	basicstyle=\small,
	frame=single,
	backgroundcolor=\color{gray30},
	language=C++,
	numbers=left
   }

 
\lstdefinestyle{Consola}
   {basicstyle=\scriptsize\bf\ttfamily,
    backgroundcolor=\color{gray30},
    frame=single,
    numbers=none
   }


\newcommand{\bigrule}{\titlerule[0.5mm]}


%Para conseguir que en las páginas en blanco no ponga cabecerass
\makeatletter
\def\clearpage{%
  \ifvmode
    \ifnum \@dbltopnum =\m@ne
      \ifdim \pagetotal <\topskip
        \hbox{}
      \fi
    \fi
  \fi
  \newpage
  \thispagestyle{empty}
  \write\m@ne{}
  \vbox{}
  \penalty -\@Mi
}
\makeatother

\usepackage{pdfpages}
\begin{document}
\begin{titlepage}
 
 
\setlength{\centeroffset}{-0.5\oddsidemargin}
\addtolength{\centeroffset}{0.5\evensidemargin}
\thispagestyle{empty}

\noindent\hspace*{\centeroffset}\begin{minipage}{\textwidth}

\centering

 \vspace{3.3cm}

\includegraphics{imagenes/logo.png} 
 \vspace{0.5cm}

% Title

{\Huge\bfseries Criptosistemas en aplicaciones de mensajería\\
}
\noindent\rule[-1ex]{\textwidth}{3pt}\\[3.5ex]
{\large\bfseries Trabajo de fin de grado en Ingeniería Informática y Matemáticas\\[4cm]}
\end{minipage}

\vspace{2.5cm}
\noindent\hspace*{\centeroffset}\begin{minipage}{\textwidth}
\centering

\textbf{Autor}\\ {Luis Tormo Fabios}\\[2.5ex]
\textbf{Director}\\
Pedro A. García Sánchez\\
\textsc{---}\\
Granada, XX de Septiembre de 2023
\end{minipage}

\vspace{\stretch{2}}

 
\end{titlepage}



%\chapter*{}
%\thispagestyle{empty}
%\cleardoublepage

%\thispagestyle{empty}

%\begin{titlepage}
 
 
\setlength{\centeroffset}{-0.5\oddsidemargin}
\addtolength{\centeroffset}{0.5\evensidemargin}
\thispagestyle{empty}

\noindent\hspace*{\centeroffset}\begin{minipage}{\textwidth}

\centering

 \vspace{3.3cm}

\includegraphics{imagenes/logo.png} 
 \vspace{0.5cm}

% Title

{\Huge\bfseries Criptosistemas en aplicaciones de mensajería\\
}
\noindent\rule[-1ex]{\textwidth}{3pt}\\[3.5ex]
{\large\bfseries Trabajo de fin de grado en Ingeniería Informática y Matemáticas\\[4cm]}
\end{minipage}

\vspace{2.5cm}
\noindent\hspace*{\centeroffset}\begin{minipage}{\textwidth}
\centering

\textbf{Autor}\\ {Luis Tormo Fabios}\\[2.5ex]
\textbf{Director}\\
Pedro A. García Sánchez\\
\textsc{---}\\
Granada, XX de Septiembre de 2023
\end{minipage}

\vspace{\stretch{2}}

 
\end{titlepage}






%\cleardoublepage
%\thispagestyle{empty}

\begin{center}
{\large\bfseries Criptosistemas en aplicaciones de mensajería}\\
\end{center}
\begin{center}
Luis Tormo Fabios\\
\end{center}

%\vspace{0.7cm}
\noindent{\textbf{Palabras clave}: Criptografía simétrica, criptografía asimétrica, cuerpos finitos, intercambio de claves, curvas elípticas, funciones hash, protocolo criptográfico.}\\

\vspace{0.7cm}
\noindent{\textbf{Resumen}}\\

En esta memoria se realiza una descripción de los criptosistemas que utilizan las aplicaciones de mensajería instantáneas más populares. Para ello empiezo con un capítulo que introduce la criptografía simétrica y asimétrica. A continuación recuerdo algunos conceptos y resultados de la teoría de aritmética modular y cuerpos finitos necesaria para entender el funcionamiento de las distintas operaciones que se realizan en los distintos criptosistemas. Después describo el funcionamiento del criptosistema simétrico \emph{AES} y posteriormente describo los criptosistemas asimétricos usados, en este caso \emph{RSA} y el intercambio de claves \emph{Diffie-Hellman}. Como este tiene una versión usando Curvas Elípticas, también introduzco la teoría necesaria acerca de estas para describir su funcionamiento. Además, también dedico un capítulo a describir las funciones hash o resumen más usadas en los distintos criptosistemas. Para concluir explico el proceso criptográfico que siguen las aplicaciones de mensajería más populares e implemento una aplicación que implementa los criptosistemas más utilizados.

\cleardoublepage


\thispagestyle{empty}


\begin{center}
{\large\bfseries Cryptography in messaging app}\\
\end{center}
\begin{center}
Luis Tormo Fabios\\
\end{center}

%\vspace{0.7cm}
\noindent{\textbf{Keywords}: Symmetric cryptography, asymmetric cryptography, finite fields, key exchange, elliptic curves, hash functions, cryptographic protocol.}\\

\vspace{0.7cm}
\noindent{\textbf{Abstract}}\\

We live in an era in which social relationships cannot be conceived without thinking about social networks and in particular messaging applications. They allow us to connect, independently of physical barriers.

In this memory, the problem that I have approached has been the development of a theoretical study of the cryptosystems used by the most used messaging applications nowadays. For this purpose, I have developed in a rigorous way the mathematical and computational tools used by messaging applications. This development includes an introductory chapter in which I make an introduction to cryptography where I explain which are the objectives that cryptosystems have to fulfill to be valid and the attacks that they may be susceptible to, following the Kerchoffs principle. Then I explain the use of symmetric and asymmetric cryptosystems in instant messaging applications and I develop more extensively what symmetric cryptosystems are as well as the \emph{ECB}, \emph{CBC}, \emph{CFB} and \emph{GCM} modes of use. Finishing with an extensive development of what are the asymmetric cryptosystems.

In the next chapter, I recall some findings of modular arithmetic and finite fields necessary to understand the operation of cryptosystems such as \emph{AES} and \emph{RSA}. About modular arithmetic I develop some findings such as what is a congruence class, what are the arithmetic functions, the Chinese Remainder Theorem, findings of Euler's Phi function in particular, Euler's Theorem, and Fermat's Small Theorem among others.
On finite fields, I focus on the Galois field and the findings of this one that allows us to work with this one in a way that at a computational level to operate with this one is not so expensive than working with other fields allowing us to reduce in a significant way the computation time and the resources used.

Once the computer and mathematical tools have been introduced, I begin to explain the most widely used cryptosystems, starting with \emph{AES}.
In this chapter, I explain how \emph{AES} works. To do this I briefly introduce its history and develop its structure explaining the rounds that it performs explaining the operations that are carried out in each of them. These operations are \emph{ByteSub}, \emph{ShiftRows}, \emph{MixColumns} and \emph{AddRoundKey}. I finished the chapter explaining how the subkeys are calculated.

In the next chapter, I explain two fundamental asymmetric cryptosystems. These are \emph{RSA} and the \emph{Diffie-Hellman} key exchange protocol.
For \emph{RSA} I give a brief historical introduction and then I explain its procedures. Finally, I explain the digital signature using \emph{RSA}, a tool widely used for message validation.
For the \emph{Diffie-Hellman} key exchange I introduce \emph{The Discrete Logarithm Problem}, a problem through which the reliability of the \emph{Diffie-Hellman} key exchange is obtained. Once I saw this I explained the key exchange. Since there is a counterpart of this key exchange using \emph{Elliptic Curves} I introduce next all the theory of elliptic curves necessary to understand its operation. Once the theory is introduced I explain the \emph{Discrete Logarithm Problem} and the operation of the \emph{Diffie-Hellman} key exchange using \emph{Elliptic Curves}.

Having seen the main symmetric and asymmetric cryptosystems used by messaging applications, I have focused on explaining \emph{hash functions}.
To do this I have introduced what hash functions are and how they are constructed using the Merkle-Damgård Construction. This is a method that allows to construction of collision-resistant hash functions. Once I have seen how to build them I have described the operation of hash functions used in messaging applications. These are \emph{SHA-0}, \emph{SHA-1} and \emph{SHA-256}.

This part concludes the explanation of the computer and mathematical tools used by messaging applications and I have proceeded to explain the protocols used by the most widely used messaging applications today. 
The first protocol I developed was the \emph{MTProto protocol}, a protocol used by the Telegram messaging application. 
The next protocol I developed was \emph{the TextSecure Protocol}. Protocol used by WhatsApp, Facebook Messenger, and Signal applications among others.
Following this I have developed the protocol used by iMessage.
The last protocol I explained was \emph{Letter Sealing} used by the Line Messenger application.

Having seen this I have documented how I have developed a simple messaging application using the tools seen throughout the document. For this, I have used the \emph{Python} programming language. For this, I have used the Python programming language due to the amount of libraries it has making the work much easier. As task manager, I used \emph{PoeThePoet}, as dependency manager I used \emph{Poetry}, and as Test Runner I used \emph{Pytest}.
To create the interface I used \emph{Tkinter}, to establish the connections between devices I used the \emph{socket library}, and to create the cryptographic functions to encrypt the messages I used the \emph{Crypto library}.

Finally, I have concluded the paper with the conclusions and possible future work that I have been drawing during the development of this paper. I have also emphasized the emergence of post-quantum cryptography since in future times it will be necessary to protect our privacy since many of the current cryptosystems are vulnerable to attacks using quantum algorithms.
\chapter*{}
\thispagestyle{empty}

\noindent\rule[-1ex]{\textwidth}{2pt}\\[4.5ex]

Yo, \textbf{Luis Tormo Fabios}, alumno del doble grado de Ingeniería Informática y Matemáticas de la \textbf{Escuela Técnica Superior
de Ingenierías Informática y de Telecomunicación y la facultad de Ciencias de la Universidad de Granada}, con DNI 80169633M, autorizo la
ubicación de la siguiente copia de mi Trabajo Fin de Grado en la biblioteca del centro para que pueda ser
consultada por las personas que lo deseen.

\vspace{6cm}

\noindent Fdo: Luis Tormo Fabios

\vspace{2cm}

\begin{flushright}
Granada a X de Septiembre de 2023.
\end{flushright}


\chapter*{}
\thispagestyle{empty}

\noindent\rule[-1ex]{\textwidth}{2pt}\\[4.5ex]

D. \textbf{Pedro A. Garcı́a Sánchez}, Profesor del Área de Álgebra del Departamento Álgebra de la Universidad de Granada.

\vspace{0.5cm}

\textbf{Informan:}

\vspace{0.5cm}

Que el presente trabajo, titulado \textit{\textbf{Criptosistemas en aplicaciones de mensajería}},
ha sido realizado bajo su supervisión por \textbf{Luis Tormo Fabios}, y autorizamos la defensa de dicho trabajo ante el tribunal
que corresponda.

\vspace{0.5cm}

Y para que conste, expiden y firman el presente informe en Granada a X de Septiembre de 2023.

\vspace{1cm}

\textbf{El director:}

\vspace{5cm}

\noindent \textbf{Pedro A. Garcı́a Sánchez}

\chapter*{Agradecimientos}
\thispagestyle{empty}

       \vspace{1cm}


En primer lugar me gustaría agredecer a mi tutor del proyecto D. Pedro por su paciencia a la hora de corregir y su ayuda constante a la hora de resolver las distintas dudas y problemas que me han ido surgiendo a lo largo del desarrollo de este trabajo.\\
A todos los profesores que me han dado clase, en especial a las señoritas Isabel y Cati, Paco, Madre Andrea y D. Francisco Rojas, por confiar y enseñarme tanto.\\
A mi familia por estar presente a lo largo de estos años y en especial a mi madrina Tere y mi prima Mirian por animarme a cursar esta carrera.\\
A mi amigo Santi, porque a pesar de que dejó de estar con nosotros, nunca se fue del todo.\\
A mis abuelos, en especial a mi abuelo Juan, ya que creo que es la única persona con más ganas que yo de que termine.\\
A mis amigos, tanto a los de siempre por estar ahí todos estos años apoyándome, como  a los nuevos que me ha dado la carrera, por ser mis compañeros de fatiga. Gracias a vosotros esto ha sido mucho más fácil de llevar.\\
A Dani, Félix y Joaquín, por estar en las buenas y en las malas. Espero en los años futuros poder devolveros todo lo bueno que me habéis dado. Y si no, al menos la mitad.\\
A mis hermanos por ser uno de los motivos por los que me levanto todos los días.\\
Y por último a mis padres, por apostar por mí cuando nadie más lo hacía y no tirar la toalla, ver en mi un potencial que no sabía que tenía y aguantarme cuando no fue fácil. Si no fuera por vosotros, esto no habría sido posible.


\frontmatter
\tableofcontents
%\listoffigures
%\listoftables
%
\mainmatter
\setlength{\parskip}{5pt}

\chapter{Introducción}
Vivimos en una época en la que no se pueden concebir las relaciones sociales sin pensar en las redes sociales y en particular las aplicaciones de mensajería. Estas nos permiten conectarnos unos con otros independientemente de las barreras físicas.
Hay aplicaciones como  \emph{WhatsApp}, \emph{Facebook Messenger} o \emph{Telegram} que tienen 2.000, 931 y 700 millones de usuarios respectivamente, lo que supone un porcentaje significativo de la población mundial que usa aplicaciones de mensajería.\\  
Debido a esta enorme cantidad de usuarios las aplicaciones tienen que garantizar su seguridad y la privacidad. Es por esto que la criptografía ha cobrado un papel fundamental en la actualidad ya que las herramientas que ofrece son las que permiten garantizar dicha seguridad y privacidad de los usuarios de las aplicaciones. 

\section{Contexto histórico}
Las aplicaciones de mensajería aparecieron en la década de 1970. Una de las primeras fue el sistema \emph{PLATO}, este era una aplicación de asistencia para la computadora basada en un sistema informático de tiempo compartido por usuarios y programadores. Fue diseñada por Bitzer con la finalidad de hacer realidad el objetivo de educar por el ordenador y entre una de sus funcionalidades había un chat para que los usuarios se comunicaran entre sí de manera local.\\ 
Entre las décadas de 1980 y 1990 apareció la aplicación \emph{TALK}, esta fue diseñada para dispositivos con sistema operativos basados en \emph{UNIX/LINUX}. Esta aplicación permitía enviar mensajes entre usuarios a través de Internet. Si bien al principio solo permitía comunicarse entre usuarios que estuvieran conectados al mismo dispositivo aunque luego se amplió la funcionalidad permitiendo el envío de mensajes entre usuarios de otros sistemas.\\
Hasta 1996 no apareció una aplicación de mensajería que se pudiera usar en otros dispositivos con distintos sistemas operativos. 
Esta fue \emph{ICQ} y supuso un antes y un después ya que fue la primera en abarcar tantos usuarios y además añadió nuevas funcionalidades. En su momento de mayor popularidad alcanzó los 38 millones de usuarios, permitiendo atisbar el potencial de las aplicaciones de mensajería como medio de comunicación.
Esta aplicación añadía nuevas funcionalidades como eran un perfil de usuario personalizable, estado de conexión, emoticonos, transferencia de contactos, transferencia de archivos y chat grupales que fueron adoptadas por las nuevas aplicaciones de mensajería manteniéndose muchas de ellas hasta hoy en día.\\
A partir de esta empezaron a surgir nuevas aplicaciones de mensajería con mayor frecuencia, estas aplicaciones usaban cada una un protocolo distinto por lo que se llevó a los usuarios a tener distintos clientes para cada aplicación.
Algunas de las aplicaciones más populares que aparecieron en esta época fueron \emph{MSN Messenger} y \emph{AIM}. La más popular fue \emph{AIM} que en 2006 tenía el control del 52 por ciento del mercado de las aplicaciones de mensajería. \emph{MSN} necesito más años para ser más popular y hasta 2005 que no alcanzó su mayor pico llegando a atraer alrededor de 330 millones de usuarios activos cada mes. Lo hizo bajo el nombre de \emph{Windows Live Messenger}.\\ 
Para compensar el creciente número de protocolo surgieron aplicaciones multiclientes que permitían soportar varios de estos protocolos, algunas de estas fueron \emph{Pidgin} o \emph{Trillian}. Ambas aplicaciones permitían comunicarse usando protocolos como \emph{MSN}, \emph{MySpaceIM}, \emph{XMPP/Jabber(Google Talk, Facebook Messenger)} y \emph{Yahoo!} entre otros.\\ 
A la vez se popularizaron las videollamadas por lo que aparecieron nuevas aplicaciones para aprovechar el nuevo nicho. Una de las primeras en aparecer fue \emph{Microsoft NetMeeting} aunque poco después apareció \emph{Skype} y se apropió de la mayoría de los usuarios de esta.\\
En 2010 los desarrolladores cambiaron de plataforma y dejaron de desarrollar aplicaciones de mensajería para ordenador para centrarse en los \emph{Smartphones}. Aparecieron aplicaciones como \emph{WhatsApp, Telegram} y \emph{Facebook Messenger} que como hemos visto al principio del capítulo, son fundamentales hoy en día y abarcan miles de millones de usuarios.

\section{Descripción del problema}
En esta memoria el problema que he abordado ha sido el desarrollo de un estudio teórico de los criptosistemas que utilizan las aplicaciones de mensajería más utilizadas en la actualidad. Para ello se desarrolla de manera rigurosa de las herramientas matemáticas e informáticas que utilizan estas. Este desarrollo abarca una introducción a la criptografía simétrica y asimétrica así como teoría de cuerpos finitos necesarios para entender adecuadamente las herramientas.
Un desarrollo en profundidad de los cifrados de bloque y del cifrado \emph{AES} en particular explicación exhaustiva de \emph{RSA} y del \emph{Problema del Logaritmo discreto} y como resultado de este, el intercambio de claves \emph{Diffie-Hellman}. Posteriormente he introducido la teoría de Curvas Elípticas necesaria para entender el análogo de las herramientas anteriores utilizando este cuerpo.
Después he introducido las \emph{funciones hash}, como construirlas usando la construcción de \emph{Merkle-Damgård} y las familias de funciones más utilizadas en las aplicaciones de mensajería actuales. A continuación he realizado una descripción como incorporan las aplicaciones de mensajería más populares los criptosistemas vistos anteriormente. Por último he desarrollado una aplicación de mensajería en la cual he aplicado lo visto previamente en la memoria \emph{Desarrollar esto más extensamente cuando tenga la aplicación}. 

\section{Técnicas utilizadas}
Las herramientas matemáticas utilizadas han sido las siguientes.
\begin{itemize}
	\item Teoría de cuerpos
	\item Teoría de números
	\item Teoría de grupos
\end{itemize}

Y las herramientas informáticas
\begin{itemize}
	\item Criptografía
	\item Algorítmica
\end{itemize}

		

%
\chapter{Criptografía y Curvas Elípticas:}
%\noindent\rule[-1ex]{\textwidth}{2pt}\\[4.5ex]

En este capítulo se introducirá la teoría sobre criptografía y curvas elípticas necesaria para entender la base detrás de los criptosistemas de las aplicaciones de mensajería.

\section{Introducción a la criptografía}
Mayormente la información de este apartado ha sido obtenida de \cite{GomezPardo2002b}
\subsection{Cifrado y secreto}

\begin{itemize}
	\item $\mathcal{M}$ el conjunto de los mensajes, textos en claro o \emph{plaintexts},
	\item $\mathcal{C}$ el conjunto de los criptogramas o \emph{cyphertexts},
	\item $\mathcal{K} \subseteq \mathcal{K}_p\times\mathcal{K}_s$ el espacio de clave o \emph{key space}
\end{itemize}
Un criptosistema viene definido por dos aplicaciones
$$E:\mathcal{K}_p\times\mathcal{M}\rightarrow\mathcal{C}$$
$$\mathcal{D}:\mathcal{K}_s\times\mathcal{C}\rightarrow\mathcal{M}$$
tales que para cualquier clave $k_p \in \mathcal{K}_p$, existe una clave $k_s$ de manera que dato cualquier mensaje $m \in \mathcal{M}$,
$$
\mathcal{D}(k_s,E(k_p,m))=m.
$$
Fijadas claves $k_p \in \mathcal{K}_p$ y sus correspondiente $k_s \in \mathcal{K}_s$ se definen las funciones de cifrado y descifrado como:
$$
	E_{k_p}:\mathcal{M}\rightarrow\mathcal{C},[E_{k_p}(m)=E(k_p,m)]
$$
$$
	D_{k_p}:\mathcal{C}\rightarrow\mathcal{M},[D_{k_s}(c)=D(k_s,c)]
$$
En la criptografía clásica, también llamada simétrica, se tiene que $\mathcal{K}_p=\mathcal{K}_s$ y $k_s = k_p = k \in \mathcal{K}$, o al menos hay métodos eficientes para conocer $k_s$ a partir de $k_p$ y viceversa. En la criptografía asimétrica, no se conocen métodos eficientes para conocer $k_s$ a partir de $k_p$.

\subsection{Objetivos de la criptografía}
\begin{itemize}
		\item \textbf{Confidencialidad:} La información solo puede ser accesible por las entidades autorizadas. 
		\item \textbf{Integridad:} La información no ha sido alterada en el envío.
		\item \textbf{Autenticidad:} La información proviene de quién afirma haberla enviado
		\item \textbf{No repudio:} El emisario de una información no puede a posteriori negar que se realizado tal envío.
\end{itemize}

\subsection{Ataques}
Se sigue el principio de Kerckhoffs el cual establece que el adversario conoce todos los detalles del criptosistema excepto la clave empleada. Los posibles ataques son:
\begin{itemize}
		\item \textbf{Criptograma} El adversario conoce el criptograma.
		\item \textbf{Mensaje Conocido} El atacante conoce parejas mensaje/criptograma cifradas con una misma clave
		\item \textbf{Mensaje escogido} El atacante puede generar criptogramas para mensajes de su elección. Una vez obtenidas dichas parejas, trata de averiguar el mensaje correspondiente a un criptograma desconocido.
		\item \textbf{Mensaje escogido-adaptativo} El atacante no solo puede generar pareas mensaje/criptograma a su elección, sino que puede hacerlo tantas veces como quiera realizando los análisis que considere oportunos
		\item \textbf{Criptograma escogido y escogido-adaptativo} Similar a los anteriores pero partiendo del criptograma, teniendo acceso a descifrar los criptogramas que desee, inicialmente o a lo largo del proceso. Lo que se busca en este ataque es la clave.
\end{itemize}
\newpage
\section{El algoritmo Rijndael AES}
El algoritmo Rijndael llamado así en honor a sus dos autores Joan Daemen y Vicent Rijmen, es un algoritmo de cifrado por bloques que fue adoptado en octubre de 2000 por el NIST(\emph{National Institute for Standards and Technology}) para su empleo en aplicaciones criptográficas no militares en sustitución del algoritmo \emph{DES} después de un proceso de más tres años en los que se buscaba un algoritmo que fuera potente, eficiente y fácil de implementar.\\
Está diseñado para manejar longitudes de clave y de bloque variables entre los 128 y los 256 bits y aunque estos sean variables, en el estándar adoptado por el Gobierno de Estados Unidos en 2001 \cite{aesUsa} establece una longitud fija de bloque de 128 bits y una longitud de clave a escoger entre 128, 192 y 256 bits.\\
La información para los siguientes apartados de AES la he obtenido de \cite{En2011}.

\subsection{Cifrados de bloque}
Son criptosistemas de clave simétrica en los que la longitud de los bloques y claves es fija.\\
Este criptosistema se define
$$
	E:\mathbb{B}^K\times\mathbb{B}^N\rightarrow \mathbb{B}^N,
$$
$$
	D:\mathbb{B}^K\times\mathbb{B}^N\rightarrow \mathbb{B}^N,
$$
Donde N es el tamaño del bloque y K es el tamaño de la clave.

\subsection{Estructura de AES}
En el algoritmo AES se define cada ronda como una composición de cuatro funciones invertibles diferentes, formando tres \emph{capas}. Estas funciones tienen un propósito específico:
\begin{itemize}
	\item \textbf{Capa de mezcla lineal:} Formada por las funciones \emph{DesplazarFila} y \emph{MezclarColumnas} y permite obtener un alto nivel de difusión a lo largo de varias rondas.
	\item \textbf{Capa no lineal:} Formada por la función \emph{ByteSub} y es la aplicación paralela de s-cajas con propiedades óptimas de no linealidad.
	\item \textbf{Capa de adición de clave:} Es un simple \emph{or-exclusivo} entre el estado intermedio y la subclave correspondiente a cada ronda.
\end{itemize}

\subsection{Elementos de AES}
AES es un algoritmo que se basa en aplicar un número determinado de rodas a un valor intermedio denominado \emph{estado} que puede ser representado por una matriz rectangular que posee cuatro filas y $N_{b}$ columnas. Análogamente la clave tiene la misma estructura, una matriz de cuatro filas y $N_{k}$.
El bloque ha cifrar o descifrar se traslada directamente byte a byte sobre la matriz de estado de columna en columna($a_{0,0}, a_{1,0}, a_{2,0}, a_{3,0}, a_{0,1} ...$)

\begin{table}[htb]
	\begin{center}
		\begin{tabular}{| l | l | l | l |}
				\hline
				$\math{a}_{0,0}$ & $\math{a}_{0,1}$ & $\math{a}_{0,2}$ & $\math{a}_{0,3}$\\ \hline
				$\math{a}_{1,0}$ & $\math{a}_{1,1}$ & $\math{a}_{1,2}$ & $\math{a}_{1,3}$\\ \hline
				$\math{a}_{2,0}$ & $\math{a}_{2,1}$ & $\math{a}_{2,2}$ & $\math{a}_{2,3}$\\ \hline
				$\math{a}_{3,0}$ & $\math{a}_{3,1}$ & $\math{a}_{3,2}$ & $\math{a}_{3,3}$\\ \hline
		\end{tabular}
		\caption{Ejemplo de matriz de estado con $N_b=4$(128 bits).}
	\end{center}
\end{table}

\begin{table}[htb]
	\begin{center}
		\begin{tabular}{| l | l | l | l |}
				\hline
				$\math{k}_{0,0}$ & $\math{k}_{0,1}$ & $\math{k}_{0,2}$ & $\math{k}_{0,3}$\\ \hline
				$\math{k}_{1,0}$ & $\math{k}_{1,1}$ & $\math{k}_{1,2}$ & $\math{k}_{1,3}$\\ \hline
				$\math{k}_{2,0}$ & $\math{k}_{2,1}$ & $\math{k}_{2,2}$ & $\math{k}_{2,3}$\\ \hline
				$\math{k}_{3,0}$ & $\math{k}_{3,1}$ & $\math{k}_{3,2}$ & $\math{k}_{3,3}$\\ \hline
		\end{tabular}
		\caption{Ejemplo de clave con $N_k=4$(128 bits).}
	\end{center}
\end{table}

En otros casos el bloque y la clave pueden ser representados como vectores de registro de 32 bits donde cada registro esta compuesto por los bytes de la columna correspondiente ordenados en orden descendiente.\\

Siendo $B$ el bloque que queremos cifrar y $S$ la matriz de estado, el algoritmo AES con $n$ quedaría:

\begin{enumerate}
	\item Calcular $K_0, K_1,...,K_n$ subclaves a partar de la clave $K$.
	\item $S\leftarrow B \oplus K_0$
	\item Para $i=1$ hasta $n$ hacer
	\begin{description}
			\item Aplicar la roda \emph{i}-ésima del algoritmo con la subclave $K_i$
	\end{description}
\end{enumerate}
Como las funciones usadas en cada ronda son invertibles, para descifrar aplicaremos las funciones inversas de las funciones usadas para cifrar en el orden opuesto.

\newpage
\begin{table}[htb]
	\begin{center}
		\begin{tabular}{| l | l | l | l |}
				\hline
				& $N_b = 4$(128 bits) & $N_b = 6$(192 bits)& $N_b = 8$(256 bits)\\ \hline
				$N_b = 4$(128 bits)& 10 & 12 & 14\\ \hline
				$N_b = 6$(128 bits)& 12 & 12 & 14\\ \hline
				$N_b = 8$(128 bits)& 14 & 14 & 14\\ \hline
		\end{tabular}
		\caption{Número de rodas en función del tamaño de la clave y bloque}
		\label{rondas_aes}
	\end{center}
\end{table}

\subsection{Las Rondas de AES}
Dado que el algoritmo AES puede aplicarse para longitudes diferentes de bloque y clave, el número de rondas es variables, como se ha visto en \ref{rondas_aes}.\\
Siendo $S$ la matriz de estado y $K_i$ la subclave correspondiente a la ronda $i$-ésima, cada ronda posee esta estructura:
\begin{enumerate}
	\item $S \leftarrow ByteSub(S)$
	\item $S \leftarrow DesplazarFila(S)$
	\item $S \leftarrow MezclarColumnas(S)$
	\item $S \leftarrow K_i \oplus S$
\end{enumerate}
En la última ronda se hacen solo los tres primeros pasos del algoritmo.

\begin{description}
	\item \textbf{ByteSub}\\
		La función \emph{ByteSub} es una sustitución no lineal que se aplica a cada byte de la matriz de estado mediante una s-caja 8\texttimes8. Se obtiene componiendo dos transformaciones:
		\begin{enumerate}
			\item Cada byte se considera como un elemento del $GF(2^8)$ generado por el polinomio irreducible $m(x)=x^8+x^4+x^3+x+1$ y es sustituido por su inversa multiplicativa quedando el valor cero inalterado. 
			\item A continuación se aplica la siguiente transformación afín en $GF(2)$ siendo $x_0, x_1,...,x_7$ los bits del byte correspondiente e $y_0, y_1,...,y_7$ los del resultado:

				\begin{equation*} 
					\begin{bmatrix} 
						y_0\\
						y_1\\
						y_2\\
						y_3\\
						y_4\\
						y_5\\
						y_6\\
						y_7\\
					\end{bmatrix}
					=
					\begin{bmatrix} % O matrices como esta de 4 x 3
						1 & 0 & 0 & 0 & 1 & 1 & 1 & 1\\
						1 & 1 & 0 & 0 & 0 & 1 & 1 & 1\\
						1 & 1 & 1 & 0 & 0 & 0 & 1 & 1\\
						1 & 1 & 1 & 1 & 0 & 0 & 0 & 1\\
						1 & 1 & 1 & 1 & 1 & 0 & 0 & 0\\
						0 & 1 & 1 & 1 & 1 & 1 & 0 & 0\\
						0 & 0 & 1 & 1 & 1 & 1 & 1 & 0\\
						0 & 0 & 0 & 1 & 1 & 1 & 1 & 1\\
					\end{bmatrix}
					\begin{bmatrix}
						x_0\\
						x_1\\
						x_2\\
						x_3\\
						x_4\\
						x_5\\
						x_6\\
						x_7\\
					\end{bmatrix}
					+
					\begin{bmatrix}
						1\\
						1\\
						0\\
						0\\
						0\\
						1\\
						1\\
						0\\
					\end{bmatrix}
			\end{equation*}
		\end{enumerate}
		La función inversa de $ByteSub$ es la aplicación inversa de la s-caja de cada byte de la matriz de estado.

	\item \textbf{DesplazarFila}\\
		Esta función desplaza a la izquierda de manera cíclica las filas de la matriz de estado. Cada fila $f_i$ se desplaza un número de posiciones $c_i$ diferente. Mientras que $c_0$ siempre es igual a cero, el resto de valores vine en función de $N_b$ como se puede ver en \ref{ciennb}.\\
		La función inversa será el desplazamiento de las filas de la matriz el mismo número de posiciones pero en el sentido contrario.

		\begin{table}[htb]
			\begin{center}
				\begin{tabular}{| l | l | l | l |}
						\hline
						$N_b$ & $c_1$ & $c_2$ & $c_3$\\ \hline
						4 & 1 & 2 & 3\\ \hline 
						6 & 1 & 2 & 3\\ \hline 
						8 & 1 & 3 & 4\\ \hline 
				\end{tabular}
				\caption{Valores de $c_i$ según el tamaño de bloque $N_b$}
				\label{ciennb}
			\end{center}
		\end{table}

		\begin{figure}[htb]
			\centering
			\includegraphics[scale=0.4]{imagenes/aesdesplazarmezclar.png} 
			\caption{Esquema de las funciones $MezclarColumnas$ y $DesplazarFila$}
			\label{desplazarymezclar}
		\end{figure}

	\item \textbf{MezclarColumnas}\\
		Durante la aplicación de esta función se considera cada columna del vector de estado se considera un polinomio cuyos coeficientes pertenecen a $GF(2^8)$ y se multiplica módulo $x^4+1$ por: $c(x)=03x^3+01x^2+01x+02$ donde 03 es el valor hexadecimal que se obtiene concatenado los coeficientes binarios del polinomio correspondiente en $GF(2^8)$, en este caso sería 00000011 y por tanto $x+1$ análogamente se haría con los demás.\\
		La inversa de $MezclarColumnas$ se obtiene multiplicando cada columna de la matriz de estado por el polinomio: $d(x)=0Bx^3+0Dx^2+09x+0E$

\end{description}

\subsection{Cálculo de las Subclaves}
Las subclaves $K_i$ se obtienen de la clave principal $K$ mediante el uso de dos funciones: una de expansión y otra de selección. Siendo $n$ el número de rondas que se van a aplicar, la función de expansión obtiene a partir del valor de $K$ una secuencia de $4(n+1)N_b$ bytes.\\
La función de selección toma consecutivamente de la secuencia obtenida bloques del mismo tamaño que la matriz de estado y los asigna a cada $K_i$.\\

Sea $K(i)$ un vector de bytes de tamaño $4N_k$ conteniendo la clave y sea $W(i)$ un vector de $N_b(n+1)$ registros de 4 bytes, siendo $n$ el número de rondas. 
La función de expansión tiene dos versiones según el valor de $N_k$:
\begin{itemize}
	\item Si $N_k<=6$:
	\begin{algorithm}
		Para i desde 0 hasta N_k-1 ha
	\end{algorithm}
	\item Si $N_k>6$:
\end{itemize}

La función \emph{Sub} devuelve el resultado de aplicar la s-caja de AES a cada uno de los bytes del registro de cuatro que se le pasa como parámetro, la función \emph{Rot} desplaza a la izquierda los bytes del registro y $RC(j)$ es una constante que se define como:
\begin{itemize}
	\item $Rc(j)=(R(j),0,0,0)$
	\item Cada $R(i)$ es el elemento de $GF(2^8)$ correspondiente al valor $x^{i-1}$ módulo $x^8+x^4+x^3+x+1$
\end{itemize}

%
\chapter{Aplicaciones de Mensajería}

Me voy a centrar en Telegram, Whatsapp y Facebook Chat, Signal y la de apple.

\section{Telegram (MTProto)}

Referencias: \cite{Miculan2021} \cite{WebProto}

\subsection{Descripción general}
MTProto 2.0 es una suite de protocolos criptográficos diseñados para implementar de manera rápida, escalable y segura intercambio de mensajes sin depositar esa responsabilidad en la seguridad del transporte debajo de dicho protocolo.
El protocolo esta subdividido en tres componentes virtuales independiente:
\begin{itemize}
	\item \textbf{Componente de alto nivel:} Define el método por el cual las consultas de la API y las respuestas se convierten en mensajes binarios. 
	\item \textbf{Capa criptográfica(autorización):} Define el método por el cual los mensajes están cifrados antes de ser enviados a través del protocolo de transporte.
	\item \textbf{Componente de transporte:} Define el método por el cual el cliente y el servidor para transmitir los mensajes sobre otro protocolo de red como HTTP, HTTPS, WS, WSS, TCP o UDP.
\end{itemize}

\includegraphics[scale=0.4]{imagenes/diagramaMTProto.jpg} 

\subsection{Resumen de los componentes}
\begin{description}
	\item \textbf{Componentes de alto nivel(Lenguajes de consulta/API RPC):}\\
Desde el punto de vista del componente de alto nivel, el cliente y el servidor intercambian mensajes dentro de una sesión.\\
La sesión se adjunta al cliente en lugar de una conexión \emph{websocket/http/https/tcp.} 
Además, cada sesión tiene asociada a clave ID de usuario mediante la cual se logra la autorización.\\ 
Pueden estar abiertas varias conexiones a un servidor, los mensajes pueden ser enviados en cualquier dirección a través de cualquiera de las conexiones.
Cuando se usa el protocolo UDP, una respuesta puede ser devuelta por una dirección de IP distinta.\\
Hay diferentes tipos de mensajes:
\begin{itemize}
		\item \textbf{LLamadas RPC(cliente-servidor):} LLamadas a los métodos de la API.
		\item \textbf{Respuestas RPC(servidor-cliente):} Resultados de las llamadas RPC.
		\item \textbf{Notificación del estado de los mensajes}
		\item \textbf{Consultas de estado de mensaje}
		\item \textbf{Mensaje multiparte o contenedor}
\end{itemize}

	\item \textbf{Autorización y Cifrado:}
	\item \textbf{Sincronización del tiempo:}
\end{description}
\subsection{Modelo de seguridad}
Los protocolos de Telegram se modelan en ProVerif \cite{ProVer}, que es un verificador criptográfico simbólico. Los protocolos y propiedades de seguridad están especificadas en una variante del \emph{$\pi$-cálculo} que es una notación desarrollada por Robin Milner, Joachim Parrow y David Walker, como un avance sobre el cálculo de sistemas comunicantes con el fin de proveer movilidad al modelado concurrente para representar procesos criptográficos y traducirlos en una teoría de Horn. \\
MTProto 2.0 sigue la siguiente regla de reducción

\lstinputlisting[
    caption=Regla Reducción 1,
    label={lst:listing-cpp},
    language=C++,
    style=CodigoC,
    ]{./ejemplos/ejemplo1.cpp}

\subsection{Protocolo MTProto 2.0}
Como se ha mencionado anteriormente MTProto es una suite cliente/servidor diseñada para acceder a servidores MTProto. Esta suite se puede dividir tres componentes principales.

\begin{itemize}
	\item \textbf{API de alto nivel:} Define como las consultas de la API y respuestas de esta son convertidas en mensajes binarios.
	\item \textbf{Componentes criptográficos y de autorización:} Definen como la aplicaciones se autentifican con el servidor y los mensajes son cifrados antes de ser transmitidos a través del protocolo de transporte.	
	\item \textbf{Componente de transporte:} Define como el cliente y el servidor realmente intercambian los mensajes por un protocolo existente como UDP, TCP, HTTP, HTTPS, etc. Cabe a destacar que también los protocolos de conexión inseguros son soportados.
	\item \textbf{Autorización:} Este módulo provee las funcionalidades para la primera autorización del cliente y del servidor. Se ejecuta solo en la primera ejecución de la aplicación.
\end{itemize}
\\

%
%\input{capitulos/04_Analisis}
%
%\input{capitulos/05_Diseno}
%
%\input{capitulos/06_Implementacion}
%
%\input{capitulos/07_Pruebas}
%
%\input{capitulos/08_Conclusiones}
%
%%\chapter{Conclusiones y Trabajos Futuros}
%
%

%%\nocite{*}

\bibliographystyle{plainurl}
\bibliography{bibliografia/library.bib}\addcontentsline{toc}{chapter}{Bibliografía}
%
%\appendix
%\input{apendices/manual_usuario/manual_usuario}
%%\input{apendices/paper/paper}
%\input{glosario/entradas_glosario}
% \addcontentsline{toc}{chapter}{Glosario}
% \printglossary
%\chapter*{}
%\thispagestyle{empty}

\end{document}
